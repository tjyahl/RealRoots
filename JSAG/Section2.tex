%Section2.tex



%%%%%%%%%%%%%%%%%%%%%%%%%%%%%%%%%%%%%%%%%%%%%%%%%%%%%%%%%%%%%%%%%%%%%%%%%%%%%%%%%
\section{Elimination}

%regular reprersentation and Stickelberger's theorem?

\textcolor{red}{Frank, how do we reference your aag paper?}

Let $\KK$ be an algebraically closed ring, and let $I\subset \KK[x_{1},\dots,x_{n}]$ be a zero-dimensional ideal. It is well-known that in that $\KK[x_{1},\dots,x_{n}]/I$ is finite dimensional, and $|\mathcal{V}(I)|< \text{dim }\left( \KK/[x_{1},
\dots, x_{n}]/I \right)$ [aag]. Given $f\in \KK[x_{1},\dots, x_{n}]$, and for each $i=1,\dots,n$, we define the multiplication map $m_{i}\in \text{End} (\KK[x_{1},\dots, x_{n}]/I)$ sending $\overline{f}$ to $\overline{x_{i}f}$. We say the induced map \[\KK[x_{1},\dots,x_{n}] \hookrightarrow \text{End }\left(\KK[x_{1},\dots, x_{n}]/I \right)\] is the \defcolor{\textit{regular representation}} of $\KK[x_{1},\dots,x_{n}]/I$.

\begin{theorem}[Stickelberger's Theorem] For each $i=1,\dots, n$, the value $\lambda \in \KK$ is an eigenvalue of the endomorphism $m_{i}$ if and only if here exists $a\in \mathcal{V}(I)$ such that $a_{i}=\lambda$.
\end{theorem}

%what we mean by elimination (minimal polynomial in Artinian ring)
Let $R$ be an Artinian ring, and let $g\in R$. By \defcolor{elimination} we refer to computing the minimal polynomial of $g$



%this is an univariate eliminant in the case ring element is variable

%rational univariate eliminant uses characteristic polynomial instead of minimal polynomial

%for zero-dimensional real systems, this gives an effective way to compute real solutions
%



{\color{red} Frank thinks we should do a Galois computation of 27 lines on a symmetric cubic surface.}

