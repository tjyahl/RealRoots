%Section2.tex



%%%%%%%%%%%%%%%%%%%%%%%%%%%%%%%%%%%%%%%%%%%%%%%%%%%%%%%%%%%%%%%%%%%%%%%%%%%%%%%%%
\section{Elimination}

%regular reprersentation and Stickelberger's theorem?

\textcolor{red}{Frank, how do we reference your aag paper?}

Let $\KK$ be an algebraically closed ring, and let $I\subset \KK[x_{1},\dots,x_{n}]$ be a zero-dimensional ideal. It is well-known that $\KK[x_{1},\dots,x_{n}]/I$ is a finite-dimensional vector space with $d:=\text{dim}\left( \KK[x_{1},
\dots, x_{n}]/I \right) = \deg(I)$, and $|\mathcal{V}(I)|< d$ [aag]. Given $f\in \KK[x_{1},\dots, x_{n}]$, and for each $i=1,\dots,n$, we define the multiplication map $m_{i}\in \text{End} (\KK[x_{1},\dots, x_{n}]/I)$ sending $\overline{f}$ to $\overline{x_{i}f}$. Hence, $x_{i}\mapsto m_{i}$ induces the injection

\begin{align*}\KK[x_{1},\dots,x_{n}]/I \hookrightarrow \text{End }\left(\KK[x_{1},\dots, x_{n}]/I \right), \end{align*}

which is the \defcolor{\textit{regular representation}} of $\KK[x_{1},\dots,x_{n}]/I$. Thus the map $m_{i}$ can be represented as a $d\times d$-matrix with respect to a basis of standard monomials.

\begin{theorem}[Stickelberger's Theorem] For each $i=1,\dots, n$, the value $\lambda \in \KK$ is an eigenvalue of the endomorphism $m_{i}$ if and only if here exists $a\in \mathcal{V}(I)$ such that $a_{i}=\lambda$.
\end{theorem}

%what we mean by elimination (minimal polynomial in Artinian ring)
In \texttt{RealRoots}, \defcolor{elimination} refers to computing the minimal polynomial of the element of an Artinian ring.
%
\begin{leftbar}
\verbatiminput{examples/minimalPolynomial1.txt}
\end{leftbar}
% 

%this is an univariate eliminant in the case ring element is variable
In the case the ring element is a variable, the minimal polynomial is the univariate eliminant.
%
\begin{leftbar}
\verbatiminput{examples/minimalPolynomial2.txt}
\end{leftbar}
%




%rational univariate eliminant uses characteristic polynomial instead of minimal polynomial

%for zero-dimensional real systems, this gives an effective way to compute real solutions
%



{\color{red} Frank thinks we should do a Galois computation of 27 lines on a symmetric cubic surface.}

