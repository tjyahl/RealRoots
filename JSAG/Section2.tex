%Section2.tex



%%%%%%%%%%%%%%%%%%%%%%%%%%%%%%%%%%%%%%%%%%%%%%%%%%%%%%%%%%%%%%%%%%%%%%%%%%%%%%%%%
\section{Elimination}\label{S:two}

Elimination is a classical symbolic method often used to solve systems of equations involving multivariate polynomials.
Geometrically, it gives the image of a variety under a polynomial map, such as a coordinate projection.
\texttt{RealRoots} implements methods for zero-dimension\-al ideals that reduce their study to that of univariate polynomials.

Let  $\Bbbk$ be a field and $I\subseteq\Bbbk[x_1,\dotsc,x_n]$ be a zero-dimension\-al ideal  with scheme $\calV(I)\subseteq\Bbbk^n$.
The Artinian ring $\defcolor{R}\vcentcolon=\Bbbk[x_1,\dotsc,x_n]/I$ is a vector space over $\Bbbk$ of dimension
$\defcolor{d}\vcentcolon=\mbox{degree}(I)$, and $|\calV(I)|\leq d$.
The ring $R$ acts on itself by multiplication.
For $f\in R$, let \defcolor{$m_f$} be the operator of multiplication by $f$: for $g\in R$, $m_f(g)=fg$.
By Stickelberger's Theorem~\cite{Cox2021}, the eigenvalues of $m_f$ are the values of $f$ at the  points of $\calV(I)$,
and the multiplicity of the eigenvalue $\lambda$ is the sum of the multiplicities in $\calV(I)$ of the inverse images,
$f^{-1}(\lambda)\cap\calV(I)$. 


The (univariate) \defcolor{eliminant} \defcolor{$g$} of $I$ with respect to $f$ is the minimal polynomial of $m_f$.
When $f$ is a variable, e.g.\ $f=x_1$, this is the monic generator of the univariate ideal $I\cap\Bbbk[x_1]$.
In general, $g$ generates the kernel of the map $\Bbbk[Z]\to R$, where $Z\mapsto f$.
The function \texttt{regularRepresentation} computes a matrix representing $m_f$ with respect to the standard basis for $I$ in
$\Bbbk[x_1,\dotsc,x_n]$, returning both.
The function \texttt{univariateEliminant} returns the minimal polynomial of $m_f$, with respect to new user-chosen variable (or the default
\texttt{Z}). 
%
\begin{leftbar}
\verbatiminput{examples/regularRep.txt}
\end{leftbar}
%
The eliminant \defcolor{$g$} of $I$ with respect to $f$ defines the image of the scheme $\calV(I)$ under $f$.
When $g$ has degree equal to the degree of $I$, then $f$ is an isomorphism and thus $g$ may be used to study the scheme
$\calV(I)$.
For example, when both are reduced, $g$ and $R$ have the same Galois groups over $\Bbbk$.
While the eliminant {\it a priori} only tells us about $f(\calV(I))$, when $f$ is \demph{separating} (injective on the
points of $\calV(I)$) it tells us more about those points.
In our running example, both $g$ and $I$ have degree eight. (For $I$, note the dimension of $m_f$ in \texttt{o17}.)
We see that $g$ is reduced and has four real roots.
%
\begin{leftbar}
\verbatiminput{examples/eliminantSturm.txt}
\end{leftbar}
%
Thus $\calV(I)$ is reduced and consists of eight points, exactly four of which are real.

A useful variant of the eliminant is a rational univariate representation of a zero-dimension\-al ideal $I$~\cite[???]{BPR}.
This is a triple $(f,\chi,\phi)$ where $f$ is a linear form that is separating for $\calV(I)$, $\chi$ is the characteristic
polynomial of $m_f$---which retains the multiplicities of points of $\calV(I)$, if not their scheme structure, and $\phi$
is a rational map $\phi\colon\Bbbk\ \to\Bbbk^n$ that restricts to a bijection between $\calV(\chi)$ and $\calV(I)$.
%
\begin{leftbar}
\verbatiminput{examples/RUR.txt}
\end{leftbar}
%
