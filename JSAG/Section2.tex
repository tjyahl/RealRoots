%Section2.tex



%%%%%%%%%%%%%%%%%%%%%%%%%%%%%%%%%%%%%%%%%%%%%%%%%%%%%%%%%%%%%%%%%%%%%%%%%%%%%%%%%
\section{Elimination}\label{S:two}

Elimination is a classical symbolic method often used to solve systems of equations.
Geometrically, it gives the image of a variety under a polynomial map, such as a coordinate projection.
\texttt{RealRoots} implements two methods for zero-dimensional ideals that reduce their study to that of univariate polynomials.

Let  $\Bbbk$ be a field and $I\subseteq\Bbbk[x_1,\dotsc,x_n]$ be a zero-dimensional ideal  with scheme $\calV(I)\subseteq\Bbbk^n$.
The Artinian ring $R=\Bbbk[x_1,\dotsc,x_n]/I$ is a vector space over $\Bbbk$ of dimension $\defcolor{d}\vcentcolon=\mbox{degree}(I)$, and
$|\calV(I)|\leq d$.
The ring $R$ acts on itself by multiplication.
For $f\in R$, let \defcolor{$m_f$} be the operator of multiplication by $f$, for $g\in R$, $m_f(g)=fg$.
By the Stickelberger's Theorem~\cite{Cox2021}, the eigenvalues of $m_f$ are the values of $f$ at the  points of $\calV(I)$,
and the multiplicity of the eigenvalue $\lambda$ is the sum of the multiplicities in $\calV(I)$ of the inverse images,
$f^{-1}(\lambda)\cap\calV(I)$. 


The (univariate) \defcolor{eliminant} of $I$ with respect to $f$ is the minimal polynomial of $m_f$.
When $f$ is a variable, e.g.\ $f=x_1$, this is the monic generator of the univariate ideal $I\cap\Bbbk[x_1]$.
In general, it is the kernel of the map $\Bbbk[Z]\to R$, where $Z\mapsto f$.
The function \texttt{regularRepresentation} computes a matrix representing $m_f$ with respect to the standard basis for $I$ in
$\Bbbk[x_1,\dotsc,x_n]$, returning both.
The function \texttt{univariateEliminant} returns the minimal polynomial of $m_f$, with respect to new user-chosen variable (or the default
\texttt{Z}).
%
\begin{leftbar}
\verbatiminput{examples/regularRep.txt}
\end{leftbar}
%

When $f$ and $I$ are represented by real polynomials, the real roots of the univariate eliminant correspond to real points of $\mathcal{V}(I)$. This allows one to study real points of zero-dimensional schemes by studying roots of the the univariate eliminant. \note{By this, the RUR is just an eliminant + the rational inverse map. We should say this explicitly rather than treating them separately.}

The computation below shows there are at least 2 real points of the scheme $\mathcal{V}(I)$. 

%
\begin{leftbar}
\verbatiminput{examples/eliminantSturm.txt}
\end{leftbar}
%

When $f$ separates the points of $\mathcal{V}(I)$, the univariate eliminant of $I$ with respect to $f$ is a univariate polynomial of degree $d$ whose roots have the same multiplicities as the points of $\mathcal{V}(I)$. That is to say, these are isomorphic schemes. In this setting, the function $f$ defines a map from $\mathcal{V}(I)$ to the roots of the roots of the univariate eliminant and there is a rational map $\phi:\CC\to\CC^n$ that restricts to an inverse of $f$. Such a tuple $(f,E,\phi)$ is known as a \defcolor{rational univariate representation} of $I$. \note{We need to settle on notation for the univariate eliminant and stop writing "univariate eliminant" every time (and replace $E$ here)}

We implement the alorithm of \note{names/citation} to compute rational univariate representations. This method has the advantage that the number of real roots and their multiplicities are preserved with an explicit isomorphism, though there is the additional cost of certifying that the chosen $f$ separates the points of $\mathcal{V}(I)$ and computing the rational map $\phi$. The method \texttt{rationalUnivariateRepresentation} outputs such a rational univariate representation tuple $(f,E,\phi)$ and is illustrated below.

%
\begin{leftbar}
\verbatiminput{examples/RUR.txt}
\end{leftbar}
%
