%Real Roots  JSAG version
%
%
%
%
%%%%%%%%%%%%%%%%%%%%%%%%%%%%%%%%%%%%%%%%%%%%%%%%%%%%%%%%%%%%%%%%%%%%%%%%%%%%%%%%%
\documentclass[12pt]{amsart}
\usepackage[margin = 1in]{geometry}
\usepackage{amsmath,amssymb,amsthm,mathtools}
\usepackage[dvipsnames]{xcolor}
\usepackage{ulem}  % strike out text
\usepackage{graphicx}
\usepackage{verbatim}
%\usepackage{caption}
%\usepackage{float}
%\usepackage{booktabs}
%\usepackage{subcaption}
%\pdfoutput=1

\usepackage{framed}

%Environments
\newtheorem{theorem}{Theorem}
\newtheorem{lemma}[theorem]{Lemma}
\newtheorem{corollary}[theorem]{Corollary}
\newtheorem{proposition}[theorem]{Proposition}
\newtheorem{algorithm}[theorem]{Algorithm}

\theoremstyle{definition}
\newtheorem{definition}[theorem]{Definition}
\newtheorem{example}[theorem]{Example}
\newtheorem{remark}[theorem]{Remark}


%Macros
\newcommand{\CC}{\mathbb{C}}
\newcommand{\RR}{\mathbb{R}}
\newcommand{\ZZ}{\mathbb{Z}}
\newcommand{\NN}{\mathbb{N}}

\DeclareMathOperator{\var}{{\rm var}}
\DeclareMathOperator{\Syl}{{\rm Syl}}

\newcommand{\defcolor}[1]{{\color{Maroon}#1}}
\newcommand{\demph}[1]{\defcolor{{\sl #1}}}
\newcommand{\note}[1]{{\color{red}(#1)}}
 

\title{Real solutions of polynomial systems}
%%%%%%%%%%%%%%%%%%%%%%%%%%%%%%%%%%%%%%%%%%%%%%%%%%%%%%%%%%%%%%%%%%%%%%%%%%%% 
\author[J.~Lopez Garcia]{Jordy Lopez Garcia} 
\address{J.~Lopez Garcia\\ 
         Department of Mathematics\\ 
         Texas A\&M University\\ 
         College Station\\ 
         Texas \ 77843\\ 
         USA} 
\email{jordy.lopez@tamu.edu}
\urladdr{https://jordylopez27.github.io/}
%%%%%%%%%%%%%%%%%%%%%%%%%%%%%%%%%%%%%%%%%%%%%%%%%%%%%%%%%%%%%%%%%%%%%%%%%%%% 
\author[K.~Maluccio]{Kelly Maluccio} 
\address{K.~Maluccio\\ 
         Department of Mathematics\\ 
         Austin Community College\\ 
         Austin\\ 
         Texas \ 78752\\ 
         USA} 
\email{kmaluccio@gmail.com}
\urladdr{https://github.com/kmaluccio}
%%%%%%%%%%%%%%%%%%%%%%%%%%%%%%%%%%%%%%%%%%%%%%%%%%%%%%%%%%%%%%%%%%%%%%%%%%%% 
\author[F.~Sottile]{Frank Sottile} 
\address{F.~Sottile\\ 
         Department of Mathematics\\ 
         Texas A\&M University\\ 
         College Station\\ 
         Texas \ 77843\\ 
         USA} 
\email{sottile@tamu.edu} 
\urladdr{https://www.math.tamu.edu/\~{}sottile} 
%%%%%%%%%%%%%%%%%%%%%%%%%%%%%%%%%%%%%%%%%%%%%%%%%%%%%%%%%%%%%%%%%%%%%%%%%%%% 
\author[T.~Yahl]{Thomas Yahl} 
\address{T.~Yahl\\ 
         Department of Mathematics\\ 
         Texas A\&M University\\ 
         College Station\\ 
         Texas \ 77843\\ 
         USA} 
\email{thomasjyahl@tamu.edu} 
\urladdr{https://tjyahl.github.io/} 
%%%%%%%%%%%%%%%%%%%%%%%%%%%%%%%%%%%%%%%%%%%%%%%%%%%%%%%%%%%%%%%%%%%%%%%%%%%%%%%%%%%%%%%%%%%%%%%%%%%%
\thanks{Research supported in part by Simons Collaboration Grant for Mathematicians 636314.}
\subjclass[2010]{}
%
%
%
\keywords{Sturm Theorem, Budan-Fourier Theorem, trace form}
\thanks{\texttt{RealRoots} version 0.1}
%%%%%%%%%%%%%%%%%%%%%%%%%%%%%%%%%%%%%%%%%%%%%%%%%%%%%%%%%%%%%%%%%%%%%%%%%%%%%%%%%%%%%%%%%%%%%%%%%%%%


%%%%%%%%%%%%%%%%%%%%%%%%%%%%%%%%%%%%%%%%%%%%%%%%%%%%%%%%%%%%%%%%%%%%%%%%%%%%%%%%%%%%%%%%%%%%%%%%%%%%
%%  Beginning
%%%%%%%%%%%%%%%%%%%%%%%%%%%%%%%%%%%%%%%%%%%%%%%%%%%%%%%%%%%%%%%%%%%%%%%%%%%%%%%%%%%%%%%%%%%%%%%%%%%%
\begin{document}


%%%%%%%%%%%%%%%%%%%%%%%%%%%%%%%%%%%%%%%%%%%%%%%%%%%%%%%%%%%%%%%%%%%%%%%%%%%%%%%%%%%%%%%%%%%%%%%%%%%%
%%   Abstract
%%%%%%%%%%%%%%%%%%%%%%%%%%%%%%%%%%%%%%%%%%%%%%%%%%%%%%%%%%%%%%%%%%%%%%%%%%%%%%%%%%%%%%%%%%%%%%%%%%%%
\begin{abstract}
 The \textit{Macaulay2} package \texttt{RealRoots} provides symbolic methods to study real solutions to systems of polynomial equations.
 It updates and expands an earlier package developed by Sottile and Grayson in 1999.
 We provide mathematical background and descriptions of the  \texttt{RealRoots} package, giving examples which illustrate some of its
 implemented methods.
 We also prove a general version of Sylvester's Theorem.
\end{abstract}
%%%%%%%%%%%%%%%%%%%%%%%%%%%%%%%%%%%%%%%%%%%%%%%%%%%%%%%%%%%%%%%%%%%%%%%%%%%%%%%%%%%%%%%%%%%%%%%%%%%%

\maketitle


%%%%%%%%%%%%%%%%%%%%%%%%%%%%%%%%%%%%%%%%%%%%%%%%%%%%%%%%%%%%%%%%%%%%%%%%%%%%%%%%%%%%%%%%%%%%%%%%%%%%
%%  Introduction  
%%%%%%%%%%%%%%%%%%%%%%%%%%%%%%%%%%%%%%%%%%%%%%%%%%%%%%%%%%%%%%%%%%%%%%%%%%%%%%%%%%%%%%%%%%%%%%%%%%%%
\section*{Introduction}

%historical background/motivation, discussion of prior results, contribution and why it's important/interesting/novel

Understanding the number of real solutions to systems of polynomials is fundamental for real algebraic geometry and for applications of
algebraic geometry.
In 1999, Grayson and Sottile \cite{So_M2} developed the \textit{Macaulay2} package \texttt{realroots} for this purpose.
That package had limited functionality and not all of its implemented methods remain compatible with modern releases of \textit{Macaulay2}.

The \textit{Macaulay2} package \texttt{RealRoots} expands and modernizes \texttt{realroots}, superseding it.
\texttt{RealRoots} implements symbolic methods for studying real solutions of polynomial systems.
This note provides some descriptions of the mathematical background and examples of methods from the package.
Its  three sections each describe related methods.

Section 1 describes methods for counting and isolating real roots of univariate polynomials, as well as methods for determining if a
polynomial is Hurwitz stable.
We could not find a proof of Sylvester's Theorem~\ref{Th:Sylvester} in complete generality in the literature, and have included a sketch for
completeness. 

Section 2 describes methods involving elimination that reduce a zero-dimensional system of multivariate polynomials to a univariate
polynomial for solving, studying the number of real solutions, or addressing other arithmetic questions, such as Galois groups.

Section 3 describes a further method for studying zero-dimensional systems, {\color{red}expand on this} based on the trace form.


%%%%%%%%%%%%%%%%%%%%%%%%%%%%%%%%%%%%%%%%%%%%%%%%%%%%%%%%%%%%%%%%%%%%%%%%%%%%%%%%%%%%%%%%%%%%%%%%%%%%
%Section1.tex

%%%%%%%%%%%%%%%%%%%%%%%%%%%%%%%%%%%%%%%%%%%%%%%%%%%%%%%%%%%%%%%%%%%%%%%%%%%%%%%%%%%%%%%%%%%%%%%%%%%%
\section{Real roots of univariate polynomials}

Let $f\in\RR[x]$ be a polynomial.
It has the form
%
 \[
   f\ =\ c_kx^{a_k}  + \dotsb + c_{1}x^{a_{1}} + c_0x^{a_0}\,,
 \]
%
where $a_k> \dotsb > a_1 > a_0$ are nonnegative integers and for $0\leq i \leq k$, $c_{i}$ is a nonzero real number.
Let $\defcolor{\var(c_0,\dotsc,c_k)}\vcentcolon=\#\{1\leq i\leq k\mid c_{i-1}c_i<0\}$ be the number of sign variations in the coefficients
of $f$.
Descartes' Rule of Signs \cite{So_Book} gives an upper bound for the number of positive real roots of $f$.

%%%%%%%%%%%%%%%%%%%%%%%%%%%%%%%%%%%%%%%%%%%%%%%%%%%%%%%%%%%%%%%%%%%%%%%%%%%%%%%%%%%%%%%%%%%%%%%%%%%%
\begin{theorem}[Descartes' Rule of Signs]
  The number, $r$,  of positive real roots of $f$, counted with multiplicity, is at most $\var(c_0,\dotsc,c_k)$ and the difference
  $\var(c_0,\dotsc,c_k)-r$ is even.
\end{theorem}
%%%%%%%%%%%%%%%%%%%%%%%%%%%%%%%%%%%%%%%%%%%%%%%%%%%%%%%%%%%%%%%%%%%%%%%%%%%%%%%%%%%%%%%%%%%%%%%%%%%%

More generally, given any sequence $c=(c_0,\dotsc,c_mk$, we compute the \demph{variation} of $c$, $\var(c)$, by counting the number of
variations in sign after removing all zero terms.
%
\begin{leftbar}
\verbatiminput{examples/variations.txt}
\end{leftbar}
%
For sequence of polynomials  $\defcolor{f_\bullet}=(f_0,\dotsc,f_k)$ in $\RR[x]$ and $a\in\RR$, \defcolor{$\var(f_\bullet,a)$} is the
variation in the sequence 
$(f_0(a),\dotsc,f_{k}(a))$. 
We extend this to $a=\pm\infty$, by taking $f(\infty)$ to be the leading coefficient of $f(x)$ and $f(-\infty)$ to be the leading
coefficient of $f(-x)$.

Given a polynomial $f\in\RR[x]$ of degree $k$, consider its sequence of derivatives,
%
 \[
   \defcolor{\delta f}\ \vcentcolon= \left(f(x),f'(x),f''(x),\dots,f^{(k)}(x)\right)\,.
 \]
%
For $a<b$ in $\RR\cup\{\pm \infty\}$, let \defcolor{$r(f,a,b)$} be the number of roots of $f$ in the interval $(a,b\hspace{.05cm}]$, counted
with multiplicity.
Budan and Fourier~\cite[Ch.\ 2]{So_Book} generalized Descartes' Rule.

%%%%%%%%%%%%%%%%%%%%%%%%%%%%%%%%%%%%%%%%%%%%%%%%%%%%%%%%%%%%%%%%%%%%%%%%%%%%%%%%%%%%%%%%%%%%%%%%%%%%
\begin{theorem}[Budan-Fourier]
  We have that $r(f,a,b)\leq \var(\delta f,a) -\var(\delta f,b)$, and the difference
  $\var(\delta f,a) -\var(\delta f,b)-r(f,a,b)$ is even. 
\end{theorem}
%%%%%%%%%%%%%%%%%%%%%%%%%%%%%%%%%%%%%%%%%%%%%%%%%%%%%%%%%%%%%%%%%%%%%%%%%%%%%%%%%%%%%%%%%%%%%%%%%%%%

Descartes' Rule is when $a=0$ and $b=\infty$.
Let us consider an example.
%
\begin{leftbar}
\verbatiminput{examples/BudanFourierBound.txt}
\end{leftbar}
%
Note that $r(f,0,\infty)=r(f,-2,1)=3$.

In contrast to these bounds, 
Sylvester's Theorem determines the actual number of real roots, and more.
The \demph{Sylvester sequence}, \defcolor{$\Syl(f,g)$} of polynomials $f,g\in\RR[x]$ is the sequence
$\left(f_0,f_1,\dotsc,f_k\right)$ of nonzero polynomials, where $f_0\vcentcolon= f, f_1 \vcentcolon= f'\cdot g$,
and for $i\geq 1$, 
%
  \[
    f_{i+1}\ \vcentcolon=\ -1\cdot \mbox{remainder}(f_{i-1},f_i)\,,
  \]
%
the negative remainder term in the division of $f_{i-1}$ by $f_i$.
The last nonzero remainder is $f_k = \gcd(f,f'g)$.
Observe that for each $1\leq i\leq k$ there exists $q_i\in\RR[x]$ such that
%
 \begin{equation}\label{Eq:divisionAlgorithm}
    f_{i-1}\ =\ q_i(x)f_i(x)-f_{i+1}(x)\,.
 \end{equation}
%
The \demph{reduced Sylvester sequence} is obtained by dividing each term in the Sylvester sequence by $f_k=\gcd(f,f'g)$.
Note that the elements $(g_0,\dotsc,g_k)$ of the reduced sequence satisfy~\eqref{Eq:divisionAlgorithm} with $f_j$ replacing $g_j$, and
we also have that $g_k=1$.

%%%%%%%%%%%%%%%%%%%%%%%%%%%%%%%%%%%%%%%%%%%%%%%%%%%%%%%%%%%%%%%%%%%%%%%%%%%%%%%%%%%%%%%%%%%%%%%%%%%%
\begin{theorem}[Sylvester]
  \label{Th:Sylvester}
  Let $f,g\in\RR[x]$ and suppose that $g_\bullet$ is the reduced Sylvester sequence of $f$ and $f'g$.
  For $a<b$ in $\RR\cup\{\pm\infty\}$ we have
  %
  \begin{multline*}
    \qquad\var(g_\bullet,a)-\var(g_\bullet,b)\ =\
    \#\{\zeta\in(a,b\hspace{.05cm}]\mid f(\zeta)=0\mbox{ and } g(\zeta)>0\}\\
      \ -\
    \#\{\zeta\in[a,b)\mid f(\zeta)=0\mbox{ and } g(\zeta)<0\}\,.\qquad
  \end{multline*}
\end{theorem}
%%%%%%%%%%%%%%%%%%%%%%%%%%%%%%%%%%%%%%%%%%%%%%%%%%%%%%%%%%%%%%%%%%%%%%%%%%%%%%%%%%%%%%%%%%%%%%%%%%%%

Observe the different roles that the endpoints $\{a,b\}$ play in this formula.

%%%%%%%%%%%%%%%%%%%%%%%%%%%%%%%%%%%%%%%%%%%%%%%%%%%%%%%%%%%%%%%%%%%%%%%%%%%%%%%%%%%%%%%%%%%%%%%%%%%%
\begin{proof}
  In~\cite{BCR}\footnote{Give a more precise reference (e.g.\ page or Theorem number)},
  Sylvester's Theorem is stated and proven when $f$ does not vanish at $a$ or at $b$,
  for the Sylvester sequence $\Syl(f,g)$, and not for the reduced Sylvester sequence $g_\bullet$.
  The proof proceeds by studying $\var(\Syl(f,g), t)$ as $t$ increases fron $a$ to $b$, noting that it may only
  change when $t$ passes a root of some element of the Sylvester sequence.
  Since multiplying a sequence by a nonzero number $f_k(t)$ does not change its variation, the proof in~\cite{BCR} also establishes the
  theorem when $f$ does not vanish at $a$ or at $b$.

  The variation $\var(g_\bullet,t)$ may only change when $t$ passes a root $\zeta\in[a,b\hspace{.05cm}]$ of some $g_i$ in
  $g_\bullet$. 
  Observe that $\zeta$ cannot be a root of two consecutive elements of $g_\bullet$.
  If it were, then by~\eqref{Eq:divisionAlgorithm} and an induction, it is a root of all elements of $g_\bullet$, and thus of $g_k=1$, which is a
  contradiction.
  Suppose that $g_i(\zeta)=0$ for some $i\geq 1$.
  By~\eqref{Eq:divisionAlgorithm} again, $g_{i-1}(x)$ and $g_{i+1}(x)$ have opposite signs for $x$ near $\zeta$ and thus
  $g_{i-1},g_i,g_{i+1}$ do not contribute to any change in $\var(g_\bullet,t)$ for $t$ near $\zeta$.
  This remains true if $\zeta=a$ and $t$ increases from $a$ or if $\zeta=b$ and $t$ approaches $b$.

  We now suppose that $g_0(\zeta)=0$ and thus $g_1(\zeta)\neq 0$.
  Then we have $f(\zeta)=0$.
  Let $m$ be the multiplicity of the root $\zeta$ of $f$.
  Then $f=(x{-}\zeta)^m h$ with $h(\zeta)\neq 0$.
  If $g(\zeta)=0$, then $(x{-}\zeta)^m$ divides $f'g$ and thus $f_k$, and so $g_0=f/f_k$ does not vanish at $\zeta$.
  Thus $g(\zeta)\neq 0$.

  Notice that $h_0\vcentcolon=f/(x{-}\zeta)^{m-1}$ and $h_1\vcentcolon=f'g/(x{-}\zeta)^{m-1}$ have the same signs for $x$ near $\zeta$
  as do $g_0$ and $g_1$.
  A computation reveals that $h_1=mhg+(x{-}\zeta)h'g$. 
 Choose $\epsilon>0$ so that $\zeta$ is the only root of any element in $g_\bullet$ lying in the interval
 $[\zeta-\epsilon,\zeta+\epsilon]$.
 We have
 %
 \[
 \begin{array}{c|c|l}
   x & h_0(x) & h_1(x)\\\hline
   \zeta-\epsilon & -\epsilon h(\zeta-\epsilon)  &
        mh(\zeta-\epsilon)g(\zeta-\epsilon) - \epsilon h'(\zeta-\epsilon)g(\zeta-\epsilon)  \rule{0pt}{13pt}\\
   \zeta     &     0    &   mh(\zeta)g(\zeta)  \rule{0pt}{13pt}\\
   \zeta+\epsilon & \epsilon h(\zeta-\epsilon)  &
        mh(\zeta-\epsilon)g(\zeta-\epsilon) + \epsilon h'(\zeta-\epsilon)g(\zeta-\epsilon)  \rule{0pt}{13pt}
 \end{array}
 \]
 %
 
 Suppose that $g(\zeta)>0$.
 Then the sign of $h_1$ on $[\zeta-\epsilon,\zeta+\epsilon]$ is opposite to the sign of $h_0(\zeta-\epsilon)$, but the same as the sign of
 $h_0(\zeta+\epsilon)$.
 Thus the variation $\var(g_\bullet,t)$ decreases by 1 as $t$ passes from $\zeta{-}\epsilon$ to $\zeta$, but is unchanged as $t$
 passes from $\zeta$ to $\zeta{+}\epsilon$.
  

 Suppose that $g(\zeta)<0$.
 Then the sign of $h_1$ on $[\zeta-\epsilon,\zeta+\epsilon]$ is the same as the sign of $h_0(\zeta-\epsilon)$, but opposite to the sign of
 $h_0(\zeta+\epsilon)$.
 Thus the variation $\var(g_\bullet,t)$ is unchanged as $t$ passes from $\zeta-\epsilon$ to $\zeta$, but increases by 1 as $t$ 
 passes from $\zeta$ to $\zeta+\epsilon$.

 Now consider the variation $\var(g_\bullet,t)$ for $t\in[a,b\hspace{.05cm}]$.
 This may only change at a number $\zeta\in[a,b]$ if $f(\zeta)=0$.
 If $g(\zeta)>0$ and $\zeta\neq b$, then it decreases by 1.
 If $g(\zeta)<0$ and $\zeta\neq a$, then it increases by 1.
 It is otherwise unchanged.
 This completes the proof.
 \end{proof}
%%%%%%%%%%%%%%%%%%%%%%%%%%%%%%%%%%%%%%%%%%%%%%%%%%%%%%%%%%%%%%%%%%%%%%%%%%%%%%%%%%%%%%%%%%%%%%%%%%%%

The \demph{Sturm sequence} of a polynomial $f\in\RR[x]$ is the Sylvester sequence $\Syl(f,1)$.
We may form the \demph{reduced Sturm sequence} as before.

%%%%%%%%%%%%%%%%%%%%%%%%%%%%%%%%%%%%%%%%%%%%%%%%%%%%%%%%%%%%%%%%%%%%%%%%%%%%%%%%%%%%%%%%%%%%%%%%%%%%
\begin{corollary}[Sturm's Theorem]
  Let $f\in\RR[x]$ and $a<b$ in $\mathbb{R}\cup\{\pm\infty\}$.
  Let $g_\bullet$ be the reduced Sturm sequence of $f$.
  Then the number of zeros of $f$ in the interval $(a,b\hspace{.05cm}]$ equals  $\var(g_\bullet,a) - \var(g_\bullet,b)$.
\end{corollary}
%%%%%%%%%%%%%%%%%%%%%%%%%%%%%%%%%%%%%%%%%%%%%%%%%%%%%%%%%%%%%%%%%%%%%%%%%%%%%%%%%%%%%%%%%%%%%%%%%%%%

Let us continue with the same polynomial $f=x(2x-3)(x^4-2)^2$ as before.
%
\begin{leftbar}
\verbatiminput{examples/Sturm.txt}
\end{leftbar}
%
Calling {\tt SturmCount(f)} without endpoints $a,b$ returns the number of real roots of $f$.
An application of Sturm's Theorem is to give \demph{isolating intervals}, which are disjoint intervals with each containing exactly one root
of $f$. 
Our implementation gives a list of pairs $\{p,q\}$ such that $(p,q]$ contains a unique root of $f$ and $q-p$ is less than a
user-provided tolerance.
The numbers $p,q$ are dyadic, lying in $\ZZ[\frac{1}{2}]$ as they are found using a binary search.
%
\begin{leftbar}
\verbatiminput{examples/realRootIsolation.txt}
\end{leftbar}
%

    
%Real root isolation


%Hurwitz stability
\theorem
Let $f(x) = \sum_{j=0}^{n}a_{j}x^{j}$ with $n\geq 1$ and $a_{n}>0$. Then $f$ is Hurwitz stable if and only if all the Hurwitz determinants $\delta_{1},\dots,\delta_{n}$ are all positive.
%
%M2 examples along the way
%



%%%%%%%%%%%%%%%%%%%%%%%%%%%%%%%%%%%%%%%%%%%%%%%%%%%%%%%%%%%%%%%%%%%%%%%%%%%%%%%%%%%%%%%%%%%%%%%%%%%%
%Section2.tex



%%%%%%%%%%%%%%%%%%%%%%%%%%%%%%%%%%%%%%%%%%%%%%%%%%%%%%%%%%%%%%%%%%%%%%%%%%%%%%%%%
\section{Elimination}

We provide two means of reducing the study of a zero-dimensional affine scheme to the study of roots of a univariate polynomial. Representing the scheme as the zero set of a multivariate polynomial ideal allows for classical symbolic methods such as univariate elimination. Elimination restricts the given ideal to a user-specified univariate subring and is readily computable, but makes no guarantee on preservation of degree or multiplicity. The rational univariate representation is a stronger version of elimination that does preserve these structures and can be used to recover coordinates of points in the original scheme, but this comes at an additional cost.

Let $I\subseteq\CC[x_1,\dotsc,x_n]$ \note{$\RR$ vs $\CC$?} be a zero-dimensional ideal with associated scheme $\mathcal{V}(I)\subseteq\CC^n$. Note the Artinian ring $\CC[x_1,\dotsc,x_n]/I$ is a vector space of dimension $d = \deg(I)$ and $|\mathcal{V}(I)|\le d$. Given an element $f\in\CC[x_1,\dotsc,x_n]/I$, the \defcolor{regular representation} of $f$ denoted by $m_f$ is the linear map on $\CC[x_1,\dotsc,x_n]/I$ sending $g\to fg$. \note{better letter than $g$?} The eigenvalues of $m_f$ are the values of $f$ evaluated at points of $\mathcal{V}(I)$, and the multiplicity of each eigenvalue is the sum of multiplicities of the corresponding points for which $f$ evaluates to this value \note{cite Stickelberger}.

The \defcolor{univariate eliminant} of $I$ with respect to $f$ is the minimal polynomial of $m_f$. When $f$ is a variable, e.g. $f=x_1$, this is the monic generator of the univariate ideal $I\cap\CC[x_1]$. Choosing a basis for the vector space $\CC[x_1,\dotsc,x_n]/I$, the regular representation is given as a $d\times d$ matrix and its minimal polynomial is easily computed. This gives an explicit way of computing univariate eliminants.

The function \texttt{regularRepresentation} takes as input a polynomial $f\in\CC[x_1,\dotsc,x_n]$ and an ideal $I\subseteq\CC[x_1,\dotsc,x_n]$ and outputs a basis for the vector space $\CC[x_1,\dotsc,x_n]/I$ and the matrix representation of the regular representation $m_f$. Note that the method \texttt{minimalPolynomial} takes various forms of input and its output is a univariate polynomial in a new variable \texttt{Z} which may be chosen by the user via the option \texttt{Variable}. The function \texttt{univariateEliminant} is an alias for the function \texttt{minimalPolynomial}. \note{maybe \texttt{minimalPolynomial} should accept the direct outout of \texttt{regularRepresentation}}\note{Should we restart the M2 I/O counter between sections?}

%
\begin{leftbar}
\verbatiminput{examples/regularRep.txt}
\end{leftbar}
%

When $f$ and $I$ are represented by real polynomials, the real roots of the univariate eliminant correspond to real points of $\mathcal{V}(I)$. This allows one to study real points of zero-dimensional schemes by studying roots of the the univariate eliminant. \note{By this, the RUR is just an eliminant + the rational inverse map. We should say this explicitly rather than treating them separately.}

The computation below shows there are at least 2 real points of the scheme $\mathcal{V}(I)$. 

%
\begin{leftbar}
\verbatiminput{examples/eliminantSturm.txt}
\end{leftbar}
%

When $f$ separates the points of $\mathcal{V}(I)$, the univariate eliminant of $I$ with respect to $f$ is a univariate polynomial of degree $d$ whose roots have the same multiplicities as the points of $\mathcal{V}(I)$. That is to say, these are isomorphic schemes. In this setting, the function $f$ defines a map from $\mathcal{V}(I)$ to the roots of the roots of the univariate eliminant and there is a rational map $\phi:\CC\to\CC^n$ that restricts to an inverse of $f$. Such a tuple $(f,E,\phi)$ is known as a \defcolor{rational univariate representation} of $I$. \note{We need to settle on notation for the univariate eliminant and stop writing "univariate eliminant" every time (and replace $E$ here)}

We implement the alorithm of \note{names/citation} to compute rational univariate representations. This method has the advantage that the number of real roots and their multiplicities are preserved with an explicit isomorphism, though there is the additional cost of certifying that the chosen $f$ separates the points of $\mathcal{V}(I)$ and computing the rational map $\phi$. The method \texttt{rationalUnivariateRepresentation} outputs such a rational univariate representation tuple $(f,E,\phi)$ and is illustrated below.

%
\begin{leftbar}
\verbatiminput{examples/RUR.txt}
\end{leftbar}
%


%%%%%%%%%%%%%%%%%%%%%%%%%%%%%%%%%%%%%%%%%%%%%%%%%%%%%%%%%%%%%%%%%%%%%%%%%%%%%%%%%%%%%%%%%%%%%%%%%%%%
%Section3.tex

%%%%%%%%%%%%%%%%%%%%%%%%%%%%%%%%%%%%%%%%%%%%%%%%%%%%%%%%%%%%%%%%%%%%%%%%%%%%%%%%%
\section{Real root location}
%trace form
%







%%%%%%%%%%%%%%%%%%%%%%%%%%%%%%%%%%%%%%%%%%%%%%%%%%%%%%%%%%%%%%%%%%%%%%%%%%%%%%%%%%%%%%%%%%%%%%%%%%%%
\begin{thebibliography}{1}

\bibitem{BCR}
J.~Bochnak, M.~Coste, and M.-F. Roy.
\newblock {\em Real algebraic geometry}, volume~36 of {\em Ergebnisse der
  Mathematik und ihrer Grenzgebiete (3)}.
\newblock Springer-Verlag, Berlin, 1998.

\bibitem{So_M2}
F.~Sottile.
\newblock From enumerative geometry to solving systems of polynomials
  equations.
\newblock In {\em Computations in algebraic geometry with {M}acaulay 2},
  volume~8 of {\em Algorithms Comput. Math.}, pages 101--129. Springer, Berlin,
  2002.

\bibitem{So_Book}
F.~Sottile.
\newblock {\em Real solutions to equations from geometry}, volume~57 of {\em
  University Lecture Series}.
\newblock American Mathematical Society, Providence, RI, 2011.

\end{thebibliography}


\end{document}
%%%%%%%%%%%%%%%%%%%%%%%%%%%%%%%%%%%%%%%%%%%%%%%%%%%%%%%%%%%%%%%%%%%%%%%%%%%%%%%%%%%%%%%%%%%%%%%%%%%%


\bibliographystyle{abbrv}
%\bibliographystyle{amsplain}
\bibliography{jsag}

\end{document}
