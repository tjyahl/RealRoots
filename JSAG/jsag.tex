%Real Roots  JSAG version
%
%
%
%
%%%%%%%%%%%%%%%%%%%%%%%%%%%%%%%%%%%%%%%%%%%%%%%%%%%%%%%%%%%%%%%%%%%%%%%%%%%%%%%%%
\documentclass[12pt]{amsart}
\usepackage[margin = 1in]{geometry}
\usepackage{amsmath,amssymb,amsthm}
\usepackage[dvipsnames]{xcolor}
\usepackage{ulem}  % strike out text
\usepackage{graphicx}
\usepackage{tikz,verbatim}
\usepackage{caption}
%\pdfoutput=1

\usepackage{framed}

%Environments
\newtheorem{theorem}{Theorem}
\newtheorem{lemma}[theorem]{Lemma}
\newtheorem{corollary}[theorem]{Corollary}
\newtheorem{proposition}[theorem]{Proposition}
\newtheorem{algorithm}[theorem]{Algorithm}

\theoremstyle{definition}
\newtheorem{definition}[theorem]{Definition}
\newtheorem{example}[theorem]{Example}
\newtheorem{remark}[theorem]{Remark}


\title{real roots}
%%%%%%%%%%%%%%%%%%%%%%%%%%%%%%%%%%%%%%%%%%%%%%%%%%%%%%%%%%%%%%%%%%%%%%%%%%%% 
\author[J.~Lopez Garcia]{Jordy Lopez Garcia} 
\address{J.~Lopez Garcia\\ 
         Department of Mathematics\\ 
         Texas A\&M University\\ 
         College Station\\ 
         Texas \ 77843\\ 
         USA} 
\email{jordy.lopez@tamu.edu}
\urladdr{https://jordylopez27.github.io/}
%%%%%%%%%%%%%%%%%%%%%%%%%%%%%%%%%%%%%%%%%%%%%%%%%%%%%%%%%%%%%%%%%%%%%%%%%%%% 
\author[K.~Maluccio]{Kelly Maluccio} 
\address{K.~Maluccio\\ 
         Department of Mathematics\\ 
         Texas A\&M University\\ 
         College Station\\ 
         Texas \ 77843\\ 
         USA} 
\email{kmaluccio@tamu.edu}
\urladdr{https://www.math.tamu.edu/~kmaluccio/}
%%%%%%%%%%%%%%%%%%%%%%%%%%%%%%%%%%%%%%%%%%%%%%%%%%%%%%%%%%%%%%%%%%%%%%%%%%%% 
\author[F.~Sottile]{Frank Sottile} 
\address{F.~Sottile\\ 
         Department of Mathematics\\ 
         Texas A\&M University\\ 
         College Station\\ 
         Texas \ 77843\\ 
         USA} 
\email{sottile@math.tamu.edu} 
\urladdr{http://www.math.tamu.edu/~sottile} 
%%%%%%%%%%%%%%%%%%%%%%%%%%%%%%%%%%%%%%%%%%%%%%%%%%%%%%%%%%%%%%%%%%%%%%%%%%%% 
\author[T.~Yahl]{Thomas Yahl} 
\address{T.~Yahl\\ 
         Department of Mathematics\\ 
         Texas A\&M University\\ 
         College Station\\ 
         Texas \ 77843\\ 
         USA} 
\email{thomasjyahl@math.tamu.edu} 
\urladdr{http://www.math.tamu.edu/~thomasjyahl} 
%%%%%%%%%%%%%%%%%%%%%%%%%%%%%%%%%%%%%%%%%%%%%%%%%%%%%%%%%%%%%%%%%%%%%%%%%%%%%%%%%% 
\thanks{grants?}

%Macros
\newcommand{\CC}{\mathbb{C}}
\newcommand{\RR}{\mathbb{R}}
\newcommand{\ZZ}{\mathbb{Z}}
\newcommand{\NN}{\mathbb{N}}

\newcommand{\defcolor}[1]{{\color{RoyalBlue}#1}}
\newcommand{\demph}[1]{\defcolor{{\sl #1}}}

%%%%%%%%%%%%%
%%Beginning%%
%%%%%%%%%%%%%
\begin{document}

%Content
%
%MinPoly/RUR
%Sturm
%Isolation
%Budan-Fourier
%TraceForm
%

%%%%%%%%%%%%
%%Abstract%%
%%%%%%%%%%%%
\begin{abstract}
The Macualay2 package \texttt{RealRoots} contains methods for studying real roots of univariate polynomials and real solutions of multivariate systems as well as @@@@@. It updates and expands the capabilities of the package ``RealRoots'' given by @@@@@. For univariate polynomials, counting roots in intervals, isolating roots symbolically, and stability. For multivariate systems there are general methods of elimination for zero dimensional systems or univariate eliminant of zero-dimensional systems, this includes the rational univariate representative. For real multivariate, we have the trace form. 
\end{abstract}
%%%%%%%%%%%%%%%%%%%%%%%%%%%%%%%%%%%%%%%%%%%%%%%%%%%%%%%%%%%%%%%%%%%%%%%%%%%%%%%%%

\maketitle


%%%%%%%%%%%%%%%%
%%Introduction%%
%%%%%%%%%%%%%%%%
\section*{Introduction}

Since the 19th century, understanding the number of real roots of a polynomial system is ubiquitious in the study of real algebraic geometry. We developed a package in the software system \textit{Macaulay2} called \texttt{RealRoots} that implements algorithms to symbollically explore real roots of polynomial systems.


%%%%%%%%%%%%%%%%%%%%%%%%%%%%%%%%%%%%%%%%%%%%%%%%%%%%%%%%%%%%%%%%%%%%%%%%%%%%%%%%%
\section{Real Univariate Systems}

Our story begins with a real univariate polynomial $f$ of the form $$c_{0}x^{a_{0}} + c_{1}x^{a_{1}} + \cdots + c_{m}x^{a_{m}},$$ where $m\in \NN\cup\{0\}, i\in\{0,1,\dots, m\}$, $c_{i} \neq 0$ are real numbers, and $a_{0} < a_{1} < \cdots < a_{m}$ are integers. Descartes' Rule of Signs [\textbf{reference here}] gives us an upper bound for the number of positive real roots of $f$.

%Descartes section
\begin{theorem}[Descartes' Rule of Signs]Let $r$ be the number of positive real roots of $f$, counting multiplicity. Then $r\leq \#\{i:1\leq i\leq m, c_{i-1}c_{i}<0\}.$ Moreover, the difference bewteen the terms above is even.\end{theorem}

By including the case when we replace $x$ by $-x$ in $f$, we can generalize Descartes' theorem to provide an upper bound for the total number of real roots of $f$.

\begin{corollary}[Descartes' Bound]The real polynomial $f$ has at most $m$ positive roots, $2m$ nonzero roots, and $2m + 1$ real roots.\end{corollary}

Although Descartes' bound is sharp and realized by $x(x^{2}-1)(x^{2}-2)\cdots(x^{2}-m)$, in many cases, such bound is too large to the actual number of real roots of a polynomial, for example, $x^{3} - x^{2} + x - 1$ only has one real root, but its Descartes' bound is 9. 

The \textit{\textcolor{red}{variation}} of a finite sequence of numbers $c$, denoted by $\textcolor{red}{var(c)}$, is the number of times its consecutive elements have opposite sign, after removing any zero terms. Now, if $F=(f_{0},f_{1},\dots,f_{k})$ is a finite sequence of real univariate polynomials, and $a\in \RR$, we define \textcolor{red}{$var(F,a)$} to be the variation of the sequence $(f_{0}(a),f_{1}(a),\dots,f_{k}(a))$. If $a=\pm\infty$, then \textcolor{red}{$var(F,\infty)$} is the variation of the leading coefficients of the polynomials $f_{i}(t)$, and \textcolor{red}{$var(F,-\infty)$} is the variation of the leading coefficients of the $f_{i}(-t)$.


If $g$ is a univariate polynomial of degree $l$, we denote by \textcolor{red}{$\delta g$} its sequence of derivatives: $$\delta g = \left(g(x),g'(x),g''(x),\dots,g^{(l)}(x)\right).$$ Let $a,b\in \RR\cup\{\pm \infty\}$ such that $a<b$, and denote by $\textcolor{red}{r(f,a,b)}$ the number of real roots of $f$ in the interval $(a,b]$. Budan and Fourier [\textbf{reference here}] developed a generalization of Descartes' Theorem.

%Budan Fourier theorem (reference, vague statement)
\begin{theorem}[Budan-Fourier] The real polynomial $f$ satisfies $$var(\delta f,a) - var(\delta f,b) \geq r(f,a,b),$$ and the difference is even.\end{theorem}
%\begin{align*}\text{var}(\delta f,a) - \text{var}(\delta f,b) \geq r(f,a,b),\end{align*}

The \textcolor{red}{Sylvester sequence} of two real univariate polynomials $f$ and $g$, denoted by \textcolor{red}{$Syl(f,g)$}, is the sequence $\left(f_{0},f_{1},\dots,f_{k}\right)$, where $f_{0} := f, f_{1} := g, f_{k} := \gcd(f,g)$, and $$-f_{i+1} = \text{remainder}(f_{i-1},f_{i}).$$

%Sylvester's theorem (reference ?, nice picture, proof outline)
\theorem[Sylvester]
The difference between the number of roots of $f$ in $(a,b\hspace{.05cm}]$ where $g$ is positive and the number of roots of $f$ in $[a,b)$ where $g$ is negative is counted by the difference in variation $$\text{var}(\text{Syl}(f,f'g),a) - \text{var}(\text{Syl}(f,f'g),b).$$

\begin{proof} By definition, we note that $Syl(f,f'g) = (f_{0},f_{1},\dots, f_{k})$ is the same as the sequence given by the Euclidean algorithm (up to signs). Next, for $i\in\{0,\dots,k\}$ set $g_{i} := f_{i}/\gcd(f,f'g)$. By construction, $g_{i}$ and $g_{i+1}$ are relatively prime. If $c$ is not a root of $f$, then $$var\left((f_{0},f_{1}),c\right) = var\left((g_{0},g_{1}),c\right),$$ because the denominators of $g_{0}$ and $g_{1}$ are the same. Similarly, for $i\in\{1,\dots,k\}$, we have $var\left((f_{i-1},f_{i},f_{i+1}),c\right) = var\left((g_{i-1},g_{i},g_{i+1}),c\right)$. Now, consider the case when $x$ passes through a root $c$ of $f$:

\begin{table}[h]
\begin{tabular}{c|ccc}
  &   & $c$ &   \\
  \hline
$f$ & $-$ & $0$ & $+$ \\
$f'g$ & $+$ & $0$ & $+$ \\
\end{tabular}
\vspace{0.3cm}
\caption*{$\textcolor{red}{g(c)>0},\textcolor{red}{f'(c_{-}>0)},\textcolor{red}{f'(c_{+}>0)}$}

\begin{tabular}{c|ccc}
  &   & $c$ &   \\
  \hline
$f$ & $-$ & $0$ & $+$ \\
$f'g$ & $-$ & $0$ & $-$ \\
\end{tabular}
\vspace{0.3cm}
\caption*{$\textcolor{red}{g(c)>0},\textcolor{blue}{f'(c_{-}<0)},\textcolor{red}{f'(c_{+}>0)}$}
\end{table}

We note that as $x$ passes through $c$, $var(\left(f_{0},f_{1}),x\right)$ increases by 1 if $g(c)>0$, and decreases by 1 if $g(c)<0$.


\end{proof}

%Sturm's theorem
\theorem[Sturm]
Let $f$ be a univariate polynomial and $a,b\in \mathbb{R}\cup\{\pm\infty\}$ with $a<b$ and $f(a),f(b)\neq 0$. Then the number of zeroes of $f$ in the interval $(a,b]$ is the difference 
\begin{align*}
\text{var}(F,a) - \text{var}(F,b),
\end{align*}
where $F$ is the Sturm sequence of $f$.
%Real root isolation

%Hurwitz stability
\theorem
Let $f(x) = \sum_{j=0}^{n}a_{j}x^{j}$ with $n\geq 1$ and $a_{n}>0$. Then $f$ is Hurwitz stable if and only if all the Hurwitz determinants $\delta_{1},\dots,\delta_{n}$ are all positive.
%
%M2 examples along the way
%

%%%%%%%%%%%%%%%%%%%%%%%%%%%%%%%%%%%%%%%%%%%%%%%%%%%%%%%%%%%%%%%%%%%%%%%%%%%%%%%%%
\section{Eliminations}
%regular reprersentation and Stickelberger's theorem?
%what we mean by elimination (minimal polynomial in Artinian ring)
%this is an univariate eliminant in the case ring element is variable
%rational univariate eliminant uses characteristic polynomial instead of minimal polynomial
%for zero-dimensional real systems, this gives an effective way to compute real solutions
%


%%%%%%%%%%%%%%%%%%%%%%%%%%%%%%%%%%%%%%%%%%%%%%%%%%%%%%%%%%%%%%%%%%%%%%%%%%%%%%%%%
\section{Real Multivariate Systems}
%trace form
%


\bibliographystyle{abbrv}
%\bibliographystyle{amsplain}
\bibliography{jsag}

\end{document}
