%Real Roots  JSAG version
%
%
%
%
%%%%%%%%%%%%%%%%%%%%%%%%%%%%%%%%%%%%%%%%%%%%%%%%%%%%%%%%%%%%%%%%%%%%%%%%%%%%%%%%%
\documentclass[12pt]{amsart}
\usepackage[margin = 1in]{geometry}
\usepackage{amsmath,amssymb,amsthm}
\usepackage[dvipsnames]{xcolor}
\usepackage{ulem}  % strike out text
\usepackage{graphicx}
\usepackage{tikz,verbatim}
\usepackage{caption}
\usepackage{float}
\usepackage{booktabs}
\usepackage{subcaption}
%\pdfoutput=1

\usepackage{framed}

%Environments
\newtheorem{theorem}{Theorem}
\newtheorem{lemma}[theorem]{Lemma}
\newtheorem{corollary}[theorem]{Corollary}
\newtheorem{proposition}[theorem]{Proposition}
\newtheorem{algorithm}[theorem]{Algorithm}

\theoremstyle{definition}
\newtheorem{definition}[theorem]{Definition}
\newtheorem{example}[theorem]{Example}
\newtheorem{remark}[theorem]{Remark}


\title{real solutions of polynomial systems}
%%%%%%%%%%%%%%%%%%%%%%%%%%%%%%%%%%%%%%%%%%%%%%%%%%%%%%%%%%%%%%%%%%%%%%%%%%%% 
\author[J.~Lopez Garcia]{Jordy Lopez Garcia} 
\address{J.~Lopez Garcia\\ 
         Department of Mathematics\\ 
         Texas A\&M University\\ 
         College Station\\ 
         Texas \ 77843\\ 
         USA} 
\email{jordy.lopez@tamu.edu}
\urladdr{https://jordylopez27.github.io/}
%%%%%%%%%%%%%%%%%%%%%%%%%%%%%%%%%%%%%%%%%%%%%%%%%%%%%%%%%%%%%%%%%%%%%%%%%%%% 
\author[K.~Maluccio]{Kelly Maluccio} 
\address{K.~Maluccio\\ 
         Department of Mathematics\\ 
         Austin Community College\\ 
         Austin\\ 
         Texas \ 78752\\ 
         USA} 
\email{kmaluccio@gmail.com}
\urladdr{https://github.com/kmaluccio}
%%%%%%%%%%%%%%%%%%%%%%%%%%%%%%%%%%%%%%%%%%%%%%%%%%%%%%%%%%%%%%%%%%%%%%%%%%%% 
\author[F.~Sottile]{Frank Sottile} 
\address{F.~Sottile\\ 
         Department of Mathematics\\ 
         Texas A\&M University\\ 
         College Station\\ 
         Texas \ 77843\\ 
         USA} 
\email{sottile@tamu.edu} 
\urladdr{https://www.math.tamu.edu/~sottile} 
%%%%%%%%%%%%%%%%%%%%%%%%%%%%%%%%%%%%%%%%%%%%%%%%%%%%%%%%%%%%%%%%%%%%%%%%%%%% 
\author[T.~Yahl]{Thomas Yahl} 
\address{T.~Yahl\\ 
         Department of Mathematics\\ 
         Texas A\&M University\\ 
         College Station\\ 
         Texas \ 77843\\ 
         USA} 
\email{thomasjyahl@tamu.edu} 
\urladdr{https://tjyahl.github.io/} 
%%%%%%%%%%%%%%%%%%%%%%%%%%%%%%%%%%%%%%%%%%%%%%%%%%%%%%%%%%%%%%%%%%%%%%%%%%%%%%%%%% 
\thanks{grants?\\
\textit{MSC2020}:\\ 
\textit{Keywords}:\\
\texttt{RealRoots} version 0.1}

%Macros
\newcommand{\CC}{\mathbb{C}}
\newcommand{\RR}{\mathbb{R}}
\newcommand{\ZZ}{\mathbb{Z}}
\newcommand{\NN}{\mathbb{N}}

\newcommand{\defcolor}[1]{{\color{RoyalBlue}#1}}
\newcommand{\demph}[1]{\defcolor{{\sl #1}}}

%%%%%%%%%%%%%
%%Beginning%%
%%%%%%%%%%%%%
\begin{document}

%Content
%
%MinPoly/RUR
%Sturm
%Isolation
%Budan-Fourier
%TraceForm
%

%%%%%%%%%%%%
%%Abstract%%
%%%%%%%%%%%%
\begin{abstract}

The \textit{Macaulay2} package \texttt{RealRoots} contains methods to symbolically explore real roots of univariate polynomials and real solutions of multivariate systems. It updates and expands the package under the same name, developed by Frank Sottile and Dan Grayson. For univariate polynomials, the package provides tools to count and isolate real roots, as well as to determine when a polynomial is \textit{Hurwitz stable}. For multivariate polynomial systems, it provides general methods of elimination to solve zero-dimensional systems, which include the \textit{rational univariate representation} of a zero-dimensional ideal. In particular, for zero-dimensional real multivariate systems, it contains the \textit{trace form} method, which counts the number of real solutions without multiplicity. We provide the algebraic geometry background behind the methods, illustrating them with examples and computations. Among the theorems implemented, we prove a general version of \textit{Sylvester's Theorem}.

%ORIGINAL:The Macualay2 package \texttt{RealRoots} contains methods for studying real roots of univariate polynomials and real solutions of multivariate systems as well as @@@@@. It updates and expands the capabilities of the package ``RealRoots'' given by @@@@@. For univariate polynomials, counting roots in intervals, isolating roots symbolically, and stability. For multivariate systems there are general methods of elimination for zero dimensional systems or univariate eliminant of zero-dimensional systems, this includes the rational univariate representative. For real multivariate, we have the trace form. 
\end{abstract}
%%%%%%%%%%%%%%%%%%%%%%%%%%%%%%%%%%%%%%%%%%%%%%%%%%%%%%%%%%%%%%%%%%%%%%%%%%%%%%%%%

\maketitle


%%%%%%%%%%%%%%%%
%%Introduction%%
%%%%%%%%%%%%%%%%
\section*{Introduction}

%historical background/motivation, discussion of prior results, contribution and why it's important/interesting/novel

Since the 19th century, understanding the number of real solutions of polynomial systems is ubiquitious in the study of real algebraic geometry. Grayson and Sottile \cite{MR1949550} developed the \textit{Macaulay2} package \texttt{RealRoots} to study enumerative aspects of real algebraic geometry. We build upon their work and implement tools to symbolically explore real solutions of polynomial systems.

The paper is divided into three sections. Section 1 is devoted to algorithms on  univariate polynomials. Among its methods are \texttt{BudanFourierBound} that gives a bound to the number of real roots of a univariate polynomial in an interval $(a,b]$, and \texttt{realRootIsolation} which isolates its real roots. The original package contained methods to count real roots of univariate polynomials by means of \textit{Sturm's Theorem}. We implement a general version of Sturm's theorem, called \textit{Sylvester's Theorem}, and provide a proof that, to the best of our knowledge, did not previously exist in literature. We also define \textit{Hurwitz stability}, and present algorithms to determine the Hurwitz stability of a univariate polynomial. 

Section 2 covers elimination methods to solve zero-dimensional systems. In particular, we expand the \texttt{eliminant} method by dividing it into the methods \texttt{univariateEliminant} and \texttt{minimalPolynomial}. We also implement algorithms to compute the \textit{rational univariate representation} of a zero-dimensional ideal \cite{cite-key}. 

Section 3 focuses on methods to count real solutions of zero-dimensional, real multivariate systems. In particular, we implement a multivariate version of Sylvester's theorem, called \texttt{traceSignature}.

%%%%%%%%%%%%%%%%%%%%%%%%%%%%%%%%%%%%%%%%%%%%%%%%%%%%%%%%%%%%%%%%%%%%%%%%%%%%%%%%%
\section{Real Univariate Polynomials}

Our story begins with a real univariate polynomial $f$ of the form $$c_{0}x^{a_{0}} + c_{1}x^{a_{1}} + \cdots + c_{m}x^{a_{m}},$$ where $i=0,1,\dots, m$, where $c_{i} \neq 0$ are real numbers, and where $a_{0} < a_{1} < \cdots < a_{m}$ are nonnegative integers. Descartes' Rule of Signs \cite{MR2830310} gives us an upper bound for the number of positive real roots of $f$.

%Descartes section
\begin{theorem}[Descartes' Rule of Signs]Let $r$ be the number of positive real roots of $f$, counting multiplicity, and let $v$ be the cardinality of the set $\{i\mid1\leq i\leq m, c_{i-1}c_{i}<0\}$. Then $r\leq v$ and $v-r\equiv 0 \mod 2$. \end{theorem}

%By including the case when we replace $x$ by $-x$ in $f$, we can generalize Descartes' theorem to provide an upper bound for the total number of real roots of $f$.

%\begin{corollary}[Descartes' Bound]The real polynomial $f$ has at most $m$ positive roots, $2m$ nonzero roots, and $2m + 1$ real roots.\end{corollary}

%In many cases, such bound is too large to the actual number of positive real roots of a polynomial, for example, $x^{3} - x^{2} + x - 1$ only has one positive real root, but it was 3 sign changes. 

Equivalently, we find $v$ by counting the number of sign changes between consecutive elements of the sequence $(c_{0},c_{1},\dots,c_{m})$. The \textit{\textcolor{Maroon}{variation}} of a finite sequence of numbers $c$, denoted by $\textcolor{Maroon}{var(c)}$, is the number of times its consecutive elements have opposite sign, after removing any zero terms. Hence $v = var\left((c_{0},c_{1},\dots,c_{m})\right)$.

If $F=(f_{0},f_{1},\dots,f_{k})$ is a finite sequence of real univariate polynomials, and $a\in \RR$, we define \textcolor{Maroon}{$var(F,a)$} to be the variation of the sequence $(f_{0}(a),f_{1}(a),\dots,f_{k}(a))$. If $a=\pm\infty$, then, for $j=0,\dots, k$, \textcolor{Maroon}{$var(F,\infty)$} is the variation of the leading coefficients of the polynomials $f_{j}(t)$, and \textcolor{Maroon}{$var(F,-\infty)$} is the variation of the leading coefficients of the polynomials $f_{j}(-t)$.

If $g$ is a univariate polynomial of degree $l$, we denote by \textcolor{Maroon}{$\delta g$} its sequence of derivatives: $$\delta g = \left(g(x),g'(x),g''(x),\dots,g^{(l)}(x)\right).$$ Let $a,b\in \RR\cup\{\pm \infty\}$ such that $a<b$, and denote by $\textcolor{Maroon}{r(f,a,b)}$ the number of real roots of $f$ in the interval $(a,b]$. Budan and Fourier \cite{MR2830310} generalized Descartes' Rule of Signs.

%Budan Fourier theorem (reference, vague statement)
\begin{theorem}[Budan-Fourier] Let $v_{a}$ denote $var(\delta f,a)$ and $v_{b}$ denote $var(\delta f,b)$. The real polynomial $f$ satisfies $$v_{a} - v_{b} \geq r(f,a,b),$$ and $v_{a} - v_{b} - r(f,a,b) \equiv 0 \mod 2$.\end{theorem}
%\begin{align*}\text{var}(\delta f,a) - \text{var}(\delta f,b) \geq r(f,a,b),\end{align*}

%We obtain Descartes' Rule of Signs using the variations $var(\delta f,0) = var\left((c_{0},c_{1},\dots,c_{m})\right)$ and $var(\delta f,\infty) = 0$. 

The previous theorems bound the number of real roots of $f$. We now describe methods to find the \textit{actual} number of its real roots. The \textcolor{Maroon}{Sylvester sequence} of two real univariate polynomials $f$ and $g$, denoted by \textcolor{Maroon}{$Syl(f,g)$}, is the sequence $\left(f_{0},f_{1},\dots,f_{k}\right)$, where $f_{0} := f, f_{1} := g, f_{k} := \gcd(f,g)$, and $$-f_{i+1} = \text{remainder}(f_{i-1},f_{i}).$$

%Sylvester's theorem (reference ?, nice picture, proof outline)
\theorem[Sylvester]
The difference between the number of roots of $f$ in $(a,b\hspace{.05cm}]$ where $g$ is positive and the number of roots of $f$ in $[a,b)$ where $g$ is negative is counted by the difference in variation $$\text{var}(\text{Syl}(f,f'g),a) - \text{var}(\text{Syl}(f,f'g),b).$$

\begin{proof} The proof when all the roots of $f$ are in $(a,b)$ is found in \cite{MR1659509}, where in their notation, a \textit{Sturm sequence} refers to our definition of a \textit{Sylvester sequence}. It is left to consider when $a$ or $b$ are roots of $f$. Set $g_{i} := f_{i}/f_{k}$, for $i=0,\dots, k$. By construction, $var((f_{0}(x),f_{1}(x)))$ (resp. $var((f_{i-1}(x),f_{i}(x),f_{i+1}(x)))$) and $var((g_{0},g_{1}))$ (resp. $var((g_{i-1}(x),g_{i}(x),g_{i+1}(x)))$) are the same when $x$ is not a root of $f$. The roots of $g_{0} = f/f_{k}$ is the number of roots of $f$ which are not roots of $g$. Moreover, $g_{i}$ and $g_{i+1}$ are relatively prime.\\

\noindent Let $c$ be a root of $g_{i}$, and denote by $c_{-}$ a number immediately to the left of $c$ (avoiding any zeros), and $c_{+}$ a number immediately to its right. If $c$ is a root of $g_{0}$, then $c$ is not a root of $g_{1}$. We obtain the following cases.

% \begin{table}[h]
% \begin{subtable}{.3\linewidth}\centering{
% \begin{tabular}{c|ccc}
%   &   & $c$ &   \\
%   \hline
% $f$ & $-$ & $0$ & $+$ \\
% $f'g$ & $-$ &  & $-$ \\
% \end{tabular}
% \vspace{0.3cm}
% \caption*{\tiny$\textcolor{red}{g(c)>0},\textcolor{blue}{f'(c_{-})<0},\textcolor{red}{f'(c_{+})>0}$}}
% \end{subtable}%
% \begin{subtable}{.3\linewidth}\centering{
% \begin{tabular}{c|ccc}
%   &   & $c$ &   \\
%   \hline
% $f$ & $-$ & $0$ & $+$ \\
% $f'g$ & $-$ &  & $-$ \\
% \end{tabular}
% \vspace{0.3cm}
% \caption*{\tiny$\textcolor{blue}{g(c)<0},\textcolor{blue}{f'(c_{-})<0},\textcolor{blue}{f'(c_{+})<0}$}}
% \end{subtable}
% \end{table}

%first row
\begin{table}[H]
\begin{subtable}{.3\linewidth}\centering{
\begin{tabular}{c|ccc}
  &   & $c=a$ &   \\
  \hline
$f$ &  & $0$ & $+$ \\
$f'g$ &  &  & $-$ \\
\end{tabular}
\vspace{0.3cm}
\caption*{\tiny$\textcolor{red}{g(c)>0},\textcolor{blue}{f'(c_{-})<0},\textcolor{red}{f'(c_{+})>0}$}}
\end{subtable}%
\begin{subtable}{.3\linewidth}\centering{
\begin{tabular}{c|ccc}
  &   & $c=b$ &   \\
  \hline
$f$ & $-$ & $0$ &  \\
$f'g$ & $+$ &  &  \\
\end{tabular}
\vspace{0.3cm}
\caption*{\tiny$\textcolor{red}{g(c)>0},\textcolor{red}{f'(c_{-})>0},\textcolor{red}{f'(c_{+})>0}$}}
\end{subtable}
\end{table}
%second row
\begin{table}[h]
\begin{subtable}{.3\linewidth}\centering{
\begin{tabular}{c|ccc}
  &   & $c=a$ &   \\
  \hline
$f$ &  & $0$ & $+$ \\
$f'g$ &  &  & $-$ \\
\end{tabular}
\vspace{0.3cm}
\caption*{\tiny$\textcolor{blue}{g(c)<0},\textcolor{blue}{f'(c_{-})<0},\textcolor{blue}{f'(c_{+})<0}$}}
\end{subtable}%
\begin{subtable}{.3\linewidth}\centering{
\begin{tabular}{c|ccc}
  &   & $c=b$ &   \\
  \hline
$f$ & $-$ & $0$ &  \\
$f'g$ & $+$ &  &  \\
\end{tabular}
\vspace{0.3cm}
\caption*{\tiny$\textcolor{red}{g(c)>0},\textcolor{blue}{f'(c_{-})<0},\textcolor{blue}{f'(c_{+})<0}$}}
\end{subtable}
\end{table}

As $x$ leaves $a$, $var((f_{0},f_{1}),x)$ increases by $1$, and as $x$ reaches $b$, $var((f_{0},f_{1}),x)$ decreases by $1$. If $c$ is a root of $g_{i}$, for $i\geq 1$, then $c$ is not a root of $g_{i-1}$ nor $g_{i+1}$. Hence $g_{i-1}(c)g_{i+1}(c)<0$. Hence, the variation of $(f_{i-1}(x),f_{i}(x),f_{i+1}(x))$, which is the same as $var((g_{i-1}(x),g_{i}(x),g_{i+1}(x)))$, remains unchanged. \end{proof}

%As $x$ passes through $c$, $var(\left(f_{0},f_{1}),x\right)$ increases by 1 if $g(c)>0$, and decreases by 1 if $g(c)<0$.

%By definition, we note that $Syl(f,f'g) = (f_{0},f_{1},\dots, f_{k})$ is the same as the sequence given by the Euclidean algorithm (up to signs). Next, for $i\in\{0,\dots,k\}$ set $g_{i} := f_{i}/\gcd(f,f'g)$. By construction, $g_{i}$ and $g_{i+1}$ are relatively prime. If $c$ is not a root of $f$, then $$var\left((f_{0},f_{1}),c\right) = var\left((g_{0},g_{1}),c\right),$$ because the denominators of $g_{0}$ and $g_{1}$ are the same. Similarly, for $i\in\{1,\dots,k\}$, we have $var\left((f_{i-1},f_{i},f_{i+1}),c\right) = var\left((g_{i-1},g_{i},g_{i+1}),c\right)$. 

%Sturm's theorem
\corollary[Sturm's Theorem]
Let $f$ be a univariate polynomial and $a,b\in \mathbb{R}\cup\{\pm\infty\}$ with $a<b$ and $f(a),f(b)\neq 0$. Then the number of zeros of $f$ in the interval $(a,b]$ is the difference 
\begin{align*}
\text{var}(F,a) - \text{var}(F,b),
\end{align*}
where $F$ is the Sturm sequence of $f$.

%Real root isolation


%Hurwitz stability
\theorem
Let $f(x) = \sum_{j=0}^{n}a_{j}x^{j}$ with $n\geq 1$ and $a_{n}>0$. Then $f$ is Hurwitz stable if and only if all the Hurwitz determinants $\delta_{1},\dots,\delta_{n}$ are all positive.
%
%M2 examples along the way
%

%%%%%%%%%%%%%%%%%%%%%%%%%%%%%%%%%%%%%%%%%%%%%%%%%%%%%%%%%%%%%%%%%%%%%%%%%%%%%%%%%
\section{Eliminations}
%regular reprersentation and Stickelberger's theorem?
%what we mean by elimination (minimal polynomial in Artinian ring)
%this is an univariate eliminant in the case ring element is variable
%rational univariate eliminant uses characteristic polynomial instead of minimal polynomial
%for zero-dimensional real systems, this gives an effective way to compute real solutions
%


%%%%%%%%%%%%%%%%%%%%%%%%%%%%%%%%%%%%%%%%%%%%%%%%%%%%%%%%%%%%%%%%%%%%%%%%%%%%%%%%%
\section{Real Multivariate Systems}
%trace form
%


\bibliographystyle{abbrv}
%\bibliographystyle{amsplain}
\bibliography{jsag}

\end{document}
