%M2-real.tex
%
% Notes for Macaulay2-real package
%
% Jordy Lopez
% Kelly Maluccio
% Thomas Yahl
% Frank Sottile
%
% 
%
%%%%%%%%%%%%%%%%%%%%%%%%%%%%%%%%%%%%%%%%%%%%%%%%%%%%%%%%%%%%%%%%%%%%%%%%%%%%%%%%%
%
%   Choose your paper size here and below
%
\documentclass[12pt]{amsart}
%\documentclass[a4paper,12pt]{amsart} 
\usepackage{amssymb}
\usepackage{bbm}        % Blackboard Bold 1
\usepackage{colordvi}
\usepackage{graphicx}
\usepackage{mathrsfs}   % for script fonts
%%%%%%%%%%% Layout %%%%%%%%%%%%%%%%%%%%%%%%%%%%%%
\usepackage{vmargin}
%\setpapersize{a4}
%
\setpapersize{USletter}
\setmargrb{2.5cm}{2.5cm}{2.5cm}{2.5cm} % --- sets all four margins.  Cool.

\hfuzz1.5pc % Don't bother to report overfull boxes if overage is < 1pc

%%%%%%%%%%%% Environments %%%%%%%%%%%%%%%%%%%%%%%%%

\numberwithin{equation}{section}
\newtheorem{theorem}{Theorem}[section]
\newtheorem{proposition}[theorem]{Proposition}
\newtheorem{lemma}[theorem]{Lemma}
\newtheorem{corollary}[theorem]{Corollary}
\theoremstyle{remark}
\newtheorem{definition}[theorem]{Definition}
\newtheorem{example}[theorem]{Example}
\newtheorem{remark}[theorem]{Remark}
%%%%%%%%%%%%%%%%%%%%%%%%%%%%%%%%%%%%%%%%%%%%%%%%%%%%%%%%%%%%%%%%%%%%%%%%%%%%%%%%%
\def\publname{\scriptsize \Red{Draft of \today}
\def\currentvolume{}
\def\currentissue{}
% for some reason it can't get the date info from here
%\issueinfo{X}{3}{January}{2003}
%\pagespan{1}{60}
\PII{}}
\copyrightinfo{}{}
%%%%%%%%%%%%%%%%%%%%%%%%%%%%%%%%%%%%%%%%%%%%%%%%%%%%%%%%%%%%%%%%%%%%%
\newcounter{FNC}[page]
\def\fauxfootnote#1{{\addtocounter{FNC}{2}\Magenta{$^\fnsymbol{FNC}$}%
     \let\thefootnote\relax\footnotetext{\Magenta{$^\fnsymbol{FNC}$#1}}}}
%%%%%%%%%%%%%%%%%%%% Macros %%%%%%%%%%%%%%%%%%%%%%%%%%%%%%%%%%%%%%%%%
\def\cprime{$'$}

\newcommand{\CC}{{\mathbb C}}
\newcommand{\NN}{{\mathbb N}}
\newcommand{\QQ}{{\mathbb Q}}
\newcommand{\RR}{{\mathbb R}}
\newcommand{\ZZ}{{\mathbb Z}}

\DeclareMathOperator{\ini}{{\rm in}}
\DeclareMathOperator{\var}{{\rm var}}
\DeclareMathOperator{\St}{{\rm St}}

%\newcommand{\defcolor}[1]{\Cyan{#1}}
\newcommand{\defcolor}[1]{\Blue{#1}}
\newcommand{\demph}[1]{\defcolor{{\sl #1}}}

\title{Notes for Macaulay2 real project}
%%%%%%%%%%%%%%%%%%%%%%%%%%%%%%%%%%%%%%%%%%%%%%%%%%%%%%%%%%%%%%%%%%%%%%%%%%%%
\author{Jordy Lopez}
%\address{Department of Mathematics\\
%         Texas A\&M University\\
%         College Station\\
%         Texas \ 77843\\
%         USA}
\email{jordy.lopez@math.tamu.edu}
%\urladdr{www.math.tamu.edu/\~{}}
%%%%%%%%%%%%%%%%%%%%%%%%%%%%%%%%%%%%%%%%%%%%%%%%%%%%%%%%%%%%%%%%%%%%%%%%%%%%
\author{Kelly Maluccio}
%\address{Department of Mathematics\\
%         Texas A\&M University\\
%         College Station\\
%         Texas \ 77843\\
%         USA}
\email{kmaluccio@math.tamu.edu}
%\urladdr{www.math.tamu.edu/\~{}}
%%%%%%%%%%%%%%%%%%%%%%%%%%%%%%%%%%%%%%%%%%%%%%%%%%%%%%%%%%%%%%%%%%%%%%%%%%%%
\author{Thomas J.\ Yahl}
%\address{Department of Mathematics\\
%         Texas A\&M University\\
%         College Station\\
%         Texas \ 77843\\
%         USA}
\email{thomasjyahl@math.tamu.edu}
%\urladdr{www.math.tamu.edu/\~{}}
%%%%%%%%%%%%%%%%%%%%%%%%%%%%%%%%%%%%%%%%%%%%%%%%%%%%%%%%%%%%%%%%%%%%%%%%%%%%
\author{Frank Sottile}
%\address{Department of Mathematics\\
%         Texas A\&M University\\
%         College Station\\
%         Texas \ 77843\\
%         USA}
\email{sottile@math.tamu.edu}
\urladdr{www.math.tamu.edu/\~{}sottile}
%%%%%%%%%%%%%%%%%%%%%%%%%%%%%%%%%%%%%%%%%%%%%%%%%%%%%%%%%%%%%%%%%%%%%%%%%%%%
\subjclass[2020]{14P99}
\keywords{}
%%%%%%%%%%%%%%%%%%%%%%%%%%%%%%%%%%%%%%%%%%%%%%%%%%%%%%%%%%%%%%%
\begin{document}

\begin{abstract}
These are notes around the project on implementing rountines to study real roots of polynomial systems in Macaulay2.
\end{abstract}

\maketitle
%%%%%%%%%%%%%%%%%%%%%%%%%%%%%%%%%%%%%%%%%%%%%%%%%%%%%%%%%%%%%%%%%%%%%%%%%%%%%%%%%
\section{Introduction}

For a sequence $c=(c_0,\dotsc,c_n)$ of real numbers, let \defcolor{$\var(c)$} be the variation (number of changes in sign) in the sequence
$c$. 
This is the number of times consecutive elements of the sequence $c$ have opposite signs, after removing any occurrences of 0 from $c$.
Given a sequence $F=(f_0(t),\dotsc,f_m(t))$ of real univariate polynomials and $a\in \RR$, let \defcolor{$\var(F,a)$} be the variation in
the sequence $(f_0(a),\dotsc,f_m(a))$.
We define \defcolor{$\var(F,\infty)$} to be the variation in the leading coefficients of the polynomials in $F$, which is $\var(F,a)$ for
$a\gg 0$ sufficiently positive, 
and set \defcolor{$\var(F,-\infty)$} to be $\var(F,a)$ for $a\ll 0$ sufficiently negative.

For a univariate polynomial $f\in\RR[t]$ of degree $m$, its derivative sequence \defcolor{$\delta f$} is the sequence of its derivatives,
$\delta f:=(f(t), f'(t), f''(t), f^{(3)}(t),\dotsc, f^{(m)}(t))$.
For numbers $a<b$ in $\RR\cup\{\pm\infty\}$, let \defcolor{$r(f,a,b)$} be the number of roots of $f$ in the interval $(a,b]$.

\begin{theorem}[Budan-Fourier Theorem]
Let $f\in\RR[t]$ be a univariate polynomial and $a<b$ be numbers in $\RR\cup\{\pm\infty\}$.
Then $\var(\delta f,a)-\var(\delta f,b)\geq r(f,a,b)$, and the difference is even.
\end{theorem}

(This implies Descartes' rule of signs for roots in $(0,\infty)$.)


Given univariate polynomials $f,g$, their \demph{Sturm sequence} \defcolor{$\St(f,g)$} is the sequence
\[
  f_0\ :=\ f\,,\
  f_1\ :=\ f'g\,,\ \ f_2\,,\, f_3\,,\ \dotsc\,,\, f_m\,,
\]
where $f_m$ is a greatest common divisor of $f$ and $f'g$, and for each $i\geq 1$,
$-f_{i+1}$ is the negative of the remainder from the Euclidean algorithm.
That is, $f_{i+1}$ is the unique polynomial of degree less than the degree of $f_i$ such that there is a polynomial $q_i$
with  $f_{i-1}=q_i f_i- f_{i+1}$.  

\begin{theorem}[Sylvester's Theorem]
Let $f,g\in\RR[x]$ be univariate polynomials and suppose that $a<b$ are numbers in  $\RR\cup\{\pm\infty\}$
such that neither is a root of $f$.
Then $\var(\St(f,g),a)-\var(\St(f,g),b)$ is the difference of the number of roots $x\in(a,b)$ of $f$ with $g(x)>0$
and those with $g(x)<0$.
\end{theorem}


\begin{theorem}[Sturm's Theorem]
Let $f\in\RR[x]$ be a univariate polynomial suppose that $a<b$ are two numbers in  $\RR\cup\{\pm\infty\}$
such that neither is a root of $f$.
Then $\var(\St(f,1),a)-\var(\St(f,1),b)$ is the number of roots of $f$ in the open interval $(a,b)$.
\end{theorem}

\begin{lemma}
Let $f=c_0+c_1t+\dotsb+c_m t^m$ be a univariate polynomial and $M:=|c_0/c_m|+\dotsb+|c_{m-1}/c_m|+1$.
Then all roots of $f$ lie in the interval $(-M,M)$.
\end{lemma}
%%%%%%%%%%%%%%%%%%%%%%%%%%%%%%%%%%%%%%%%%%%%%%%%%%%%%%%%%%%%%%%%%%%%%%%%%%%%%%%%%
\begin{proof}
Let $|x|>M$.
If $b_i:=c_i/c_m$, then $f(x)=c_m x^m(b_0 x^{-m}+\dotsb+b_{m-1}x^{-1} +1)$.
As
\[
   |b_0 x^{-m}+\dotsb+b_{m-1} x^{-1}|\ <\
   (|b_0|+\dotsb+|b_{m-1}|)M^{-1}\ <\ 1\,,
\]
we see that $f(x)\neq 0$.
\end{proof}

This gives the start of a bisection algorithm for locating the roots of $f$.



   
%%%%%%%%%%%%%%%%%%%%%%%%%%%%%%%%%%%%%%%%%%%%%%%%%%%%%%%%%%%%%%%%%%%%%%%%%%%%%%%%%
\bibliographystyle{amsplain}    
\bibliography{bibl}    





%@
\end{document}    

